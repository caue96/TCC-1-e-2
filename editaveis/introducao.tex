\part{Introdução}

\chapter[Introdução]{Introdução}



\section{Definição do Escopo}



\section{Motivação}



\section{Justificativa}



\section{Objetivos}

Os objetivos são subdivididos em gerais e específicos onde gerais trazemos o propósito completo e específicos expomos a visão mais detalhada destes objetivos.

\subsection{Objetivo Geral}

O objetivo geral do trabalho a ser realizado é a criação de um jogo digital como produto de software, com foco no auxílio do ensino-aprendizagem das heurísticas de Nielsen.

\subsection{Objetivos Específicos}
\begin{itemize}
\item \textbf{OE01} - Desenvolver um estudo sobre os conceitos envolvidos em jogos digitais que auxiliam no processo de ensino-aprendizagem.
\item \textbf{OE02} - Aplicar as heurísticas de usabilidades no jogo digital a ser desenvolvido.
\item \textbf{OE03} - Aplicar conceitos de engenharia de software acumulados durante o curso no progresso de desenvolvimento do jogo.
\item \textbf{OE04} - Desenvolver o jogo para auxiliar estudantes no processo de ensino-aprendizagem das heurísticas de Nielsen.
\end{itemize}

\section{Plano de Estudo}



\section{Estrutura do Trabalho}

