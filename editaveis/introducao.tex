\part{Introdução}

\chapter[Introdução]{Introdução}

A interação entre humanos e computadores vem crescendo a cada década que se passa. Com a constante evolução das tecnologias, os seres humanos tem ficado cada vez mais dependentes delas e tem passado cada vez mais tempo utilizando-as. Em contrapartida, a produtividade das pessoas tem aumentado consideravelmente com o passar dos anos devido a automação e agilidade cada vez maior que as tecnologias tem trazido. \cite{prado2006arquitetura}

Dentro do contexto da engenharia de \textit{software}, a área de estudos sobre a interação humano-computador (IHC)

\section{Definição do Escopo}

Nesta etapa foi definido o processo de pesquisa e levantamento de requisitos do desenvolvimento do jogo, alinhando as heurísticas  escolhidas do modelo ADDIE com os objetivos definidos do produto.
Os stakeholders do projeto são estudantes de Engenharia de Software que buscam capacitação em estudos de IHC na área de usabilidade no uso das heurísticas de Nilsen. 
Para a produção do jogo foi definido uma Estrutura Analítica do Projeto(EAP) que subdivide em TCC1 e TCC2. Na figura 1 encontra-se a EAP definida em estrutura híbrida que possui as etapas e as sub tarefas contidas dentro dela. \cite{boletimdogerenciamento}


\section{Motivação}

Este trabalho visa o desenvolvimento de um jogo que auxilie o processo de ensino e aprendizagem sobre as heurísticas de Nielsen. Jogos estimulam o cérebro das pessoas para resolver problemas e alcançar conquistas que trazem um bem estar quando alcançado o objetivo do usuário (FALTA REFERÊNCIA). Nesse sentido do jogo estimular pessoas, foi pensado em desenvolver um para reforçar o processo de ensino-aprendizagem. O conteúdo do jogo está inserido dentro da área de IHC e o jogo poderá ser usado para auxiliar nesse processo de ensino em sala de aula e em educação a distância (EAD).

\section{Justificativa}

Os jogos já existentes que tem como objetivo auxiliar esse processo de ensino-aprendizagem na temática das heurísticas de Nielsen são os seguinte: (CITAR OS JOGOS AQUI). Esses jogos possuem algumas falhar e pontos de melhoria quando pensamos em um jogo com objetivo de auxiliar os alunos nesse processo.

\section{Objetivos}

Os objetivos são subdivididos em gerais e específicos onde gerais trazemos o propósito completo e específicos expomos a visão mais detalhada destes objetivos.

\subsection{Objetivo Geral}

O objetivo geral do trabalho a ser realizado é a criação de um jogo digital como produto de software, com foco no auxílio do ensino-aprendizagem das heurísticas de Nielsen.

\subsection{Objetivos Específicos}
\begin{itemize}
\item \textbf{OE01} - Desenvolver um estudo sobre os conceitos envolvidos em jogos digitais que auxiliam no processo de ensino-aprendizagem.
\item \textbf{OE02} - Aplicar as heurísticas de usabilidades no jogo digital a ser desenvolvido.
\item \textbf{OE03} - Aplicar conceitos de engenharia de software acumulados durante o curso no progresso de desenvolvimento do jogo.
\item \textbf{OE04} - Desenvolver o jogo aplicando boas práticas de programação e de processo de desenvolvimento de \textit{software}.
\end{itemize}

\section{Plano de Estudo}



\section{Estrutura do Trabalho}

